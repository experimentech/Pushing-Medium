\section*{Skeletons in Field Analysis}

\subsection*{Definition}
In computational physics and image analysis, a \emph{skeleton} is a reduced, one--pixel--wide (or one--voxel--wide) representation of the essential structure of a scalar field. It traces the ``ridge lines'' or ``spines'' that follow maxima, ridges, or other topologically significant features.

\subsection*{Relevance to Physics Modelling}
In a refractive--index field $n(\mathbf{r})$, skeletons can:
\begin{itemize}
    \item Trace maxima, ridges, or filamentary structures where gradients are strong.
    \item Provide a topology--preserving simplification of the field, retaining connectivity and geometry without the surrounding bulk.
    \item Serve as a basis for quantifying structure (lengths, branching, curvature) and for seeding flow lines or ray bundles along physically relevant paths.
\end{itemize}

\subsection*{Computation Methods}
Common approaches to skeleton extraction include:
\begin{enumerate}
    \item \textbf{Morphological thinning} --- iteratively eroding a binary mask until only the medial axis remains.
    \item \textbf{Medial axis transform} --- identifying points equidistant to at least two boundaries.
    \item \textbf{Hessian--based ridge detection} --- computing the Hessian matrix
    

\[
    H_{ij} = \frac{\partial^2 n}{\partial x_i \partial x_j}
    \]


    and finding eigenvectors/eigenvalues that indicate ridge--like curvature, then tracing along them.
    \item \textbf{Topological persistence} --- using contour trees or Morse--Smale complexes to extract stable ridges.
\end{enumerate}

For refractive--index fields, the Hessian--based method is often ideal:
\begin{itemize}
    \item Compute $\nabla n$ and $H_{ij}$.
    \item Identify points where the gradient is orthogonal to a principal curvature direction and curvature is negative along that direction.
    \item Trace these points to form continuous skeleton lines.
\end{itemize}

\subsection*{Application in the Pushing--Medium Model}
In the pushing--medium framework:
\begin{itemize}
    \item Skeletons reveal the ``bones'' of the refractive--index landscape, highlighting coherent structures such as beam paths, valve edges, or trap boundaries.
    \item They can be overlaid with flow maps of $\mathbf{u}_g(\mathbf{r},t)$ to study the interplay between static refractive structures and dynamic transport.
    \item This combination helps identify whether skeleton ridges act as highways (aligned with flow) or barriers (orthogonal to flow), and how dynamic changes in $\mathbf{u}_g$ shift, merge, or break these structures.
\end{itemize}

