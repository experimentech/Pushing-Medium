\documentclass[10pt,a4paper]{article}
\usepackage{amsmath,amssymb}
\usepackage{multicol}
\usepackage[margin=1in]{geometry}

\begin{document}

\begin{multicols}{2}

% ---------------- Skeletons ----------------
\section*{Skeletons in Field Analysis}

\subsection*{Definition}
In computational physics and image analysis, a \emph{skeleton} is a reduced, one--pixel--wide (or one--voxel--wide) representation of the essential structure of a scalar field. It traces the ``ridge lines'' or ``spines'' that follow maxima, ridges, or other topologically significant features.

\subsection*{Relevance to Physics Modelling}
In a refractive--index field $n(\mathbf{r})$, skeletons can:
\begin{itemize}
    \item Trace maxima, ridges, or filamentary structures where gradients are strong.
    \item Provide a topology--preserving simplification of the field, retaining connectivity and geometry without the surrounding bulk.
    \item Serve as a basis for quantifying structure (lengths, branching, curvature) and for seeding flow lines or ray bundles along physically relevant paths.
\end{itemize}

\subsection*{Computation Methods}
Common approaches to skeleton extraction include:
\begin{enumerate}
    \item \textbf{Morphological thinning} --- iteratively eroding a binary mask until only the medial axis remains.
    \item \textbf{Medial axis transform} --- identifying points equidistant to at least two boundaries.
    \item \textbf{Hessian--based ridge detection} --- computing the Hessian matrix
    

\[
    H_{ij} = \frac{\partial^2 n}{\partial x_i \partial x_j}
    \]


    and finding eigenvectors/eigenvalues that indicate ridge--like curvature, then tracing along them.
    \item \textbf{Topological persistence} --- using contour trees or Morse--Smale complexes to extract stable ridges.
\end{enumerate}

For refractive--index fields, the Hessian--based method is often ideal:
\begin{itemize}
    \item Compute $\nabla n$ and $H_{ij}$.
    \item Identify points where the gradient is orthogonal to a principal curvature direction and curvature is negative along that direction.
    \item Trace these points to form continuous skeleton lines.
\end{itemize}

\subsection*{Application in the Pushing--Medium Model}
In the pushing--medium framework:
\begin{itemize}
    \item Skeletons reveal the ``bones'' of the refractive--index landscape, highlighting coherent structures such as beam paths, valve edges, or trap boundaries.
    \item They can be overlaid with flow maps of $\mathbf{u}_g(\mathbf{r},t)$ to study the interplay between static refractive structures and dynamic transport.
    \item This combination helps identify whether skeleton ridges act as highways (aligned with flow) or barriers (orthogonal to flow), and how dynamic changes in $\mathbf{u}_g$ shift, merge, or break these structures.
\end{itemize}

\columnbreak

% ---------------- Flow Map Modelling ----------------
\section*{Flow Map Modelling}

\subsection*{Definition}
A \emph{flow map} is a representation of how points in space are transported over time by a vector field.  
Formally, given a velocity field $\mathbf{u}(\mathbf{r},t)$, the flow map over a time interval $T$ is the mapping


\[
\Phi_T : \mathbf{x}_0 \mapsto \mathbf{x}(T; \mathbf{x}_0)
\]


where $\mathbf{x}(T; \mathbf{x}_0)$ is the position at time $T$ of a particle that started at $\mathbf{x}_0$ at $t=0$.

\subsection*{Relevance to Physics Modelling}
In the pushing--medium framework, the vector field $\mathbf{u}_g(\mathbf{r},t)$ represents the ``pushing'' velocity of the medium.  
Flow maps are used to:
\begin{itemize}
    \item Determine where material points, rays, or test particles are transported over a given time.
    \item Reveal coherent structures such as attracting/repelling regions, transport barriers, and mixing zones.
    \item Provide a Lagrangian perspective on transport, complementing the Eulerian view of the instantaneous field.
\end{itemize}

\subsection*{Construction}
To build a flow map:
\begin{enumerate}
    \item \textbf{Define the velocity field} $\mathbf{u}_g(\mathbf{r},t)$, either steady or time--dependent.
    \item \textbf{Integrate trajectories} from a grid of initial positions $\mathbf{x}_0$ over a fixed time $T$:
    

\[
    \mathbf{x}(T; \mathbf{x}_0) = \mathbf{x}_0 + \int_0^T \mathbf{u}_g(\mathbf{x}(t), t) \, dt
    \]


    \item \textbf{Store the mapping} $\Phi_T(\mathbf{x}_0)$ for later analysis.
    \item \textbf{Analyse the map}:
    \begin{itemize}
        \item Compute the Jacobian $D\Phi_T$ to obtain stretching rates.
        \item Derive Finite--Time Lyapunov Exponents (FTLE) to visualise attracting and repelling structures.
        \item Compare with static field features (e.g., skeletons) to study structure--transport interplay.
    \end{itemize}
\end{enumerate}

\subsection*{Application in the Pushing--Medium Model}
In the pushing--medium context:
\begin{itemize}
    \item Flow maps show how $\mathbf{u}_g$ carries particles or rays through the refractive--index landscape $n(\mathbf{r},t)$.
    \item Overlaying flow maps with skeletons of $n(\mathbf{r})$ reveals whether static refractive structures act as highways (aligned with flow) or barriers (orthogonal to flow).
    \item Time--dependent $\mathbf{u}_g$ fields can be engineered to shift, merge, or break skeleton structures, enabling dynamic control of transport (e.g., valves, traps, beam steering).
\end{itemize}

\end{multicols}

\end{document}

