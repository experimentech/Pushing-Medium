\section*{Flow Map Modelling}

\subsection*{Definition}
A \emph{flow map} is a representation of how points in space are transported over time by a vector field.  
Formally, given a velocity field $\mathbf{u}(\mathbf{r},t)$, the flow map over a time interval $T$ is the mapping


\[
\Phi_T : \mathbf{x}_0 \mapsto \mathbf{x}(T; \mathbf{x}_0)
\]


where $\mathbf{x}(T; \mathbf{x}_0)$ is the position at time $T$ of a particle that started at $\mathbf{x}_0$ at $t=0$.

\subsection*{Relevance to Physics Modelling}
In the pushing--medium framework, the vector field $\mathbf{u}_g(\mathbf{r},t)$ represents the ``pushing'' velocity of the medium.  
Flow maps are used to:
\begin{itemize}
    \item Determine where material points, rays, or test particles are transported over a given time.
    \item Reveal coherent structures such as attracting/repelling regions, transport barriers, and mixing zones.
    \item Provide a Lagrangian perspective on transport, complementing the Eulerian view of the instantaneous field.
\end{itemize}

\subsection*{Construction}
To build a flow map:
\begin{enumerate}
    \item \textbf{Define the velocity field} $\mathbf{u}_g(\mathbf{r},t)$, either steady or time--dependent.
    \item \textbf{Integrate trajectories} from a grid of initial positions $\mathbf{x}_0$ over a fixed time $T$:
    

\[
    \mathbf{x}(T; \mathbf{x}_0) = \mathbf{x}_0 + \int_0^T \mathbf{u}_g(\mathbf{x}(t), t) \, dt
    \]


    \item \textbf{Store the mapping} $\Phi_T(\mathbf{x}_0)$ for later analysis.
    \item \textbf{Analyse the map}:
    \begin{itemize}
        \item Compute the Jacobian $D\Phi_T$ to obtain stretching rates.
        \item Derive Finite--Time Lyapunov Exponents (FTLE) to visualise attracting and repelling structures.
        \item Compare with static field features (e.g., skeletons) to study structure--transport interplay.
    \end{itemize}
\end{enumerate}

\subsection*{Application in the Pushing--Medium Model}
In the pushing--medium context:
\begin{itemize}
    \item Flow maps show how $\mathbf{u}_g$ carries particles or rays through the refractive--index landscape $n(\mathbf{r},t)$.
    \item Overlaying flow maps with skeletons of $n(\mathbf{r})$ reveals whether static refractive structures act as highways (aligned with flow) or barriers (orthogonal to flow).
    \item Time--dependent $\mathbf{u}_g$ fields can be engineered to shift, merge, or break skeleton structures, enabling dynamic control of transport (e.g., valves, traps, beam steering).
\end{itemize}

