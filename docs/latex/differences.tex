\usepackage{tabularx} % in your preamble
\renewcommand{\arraystretch}{1.3} % more breathing room

\begin{table}[h!]
\centering
\begin{tabularx}{\textwidth}{|p{3cm}|X|X|}
\hline
\textbf{Aspect} & \textbf{Pushing‑Medium Model} & \textbf{General Relativity} \\
\hline
\textbf{Underlying picture} &
Flat, fixed Euclidean background. Gravity‑like effects come from a medium with spatially/temporally varying refractive index $n(\mathbf{r},t)$ and optional flow field $\mathbf{u}_g$. &
Spacetime itself is dynamic and curved. Matter/energy changes curvature; curvature dictates motion. \\
\hline
\textbf{Core variables} &
Scalar $n(\mathbf{r},t)$ + vector $\mathbf{u}_g(\mathbf{r},t)$ fields. Optional explicit wave perturbations. &
Metric tensor $g_{\mu\nu}(x^\alpha)$ encodes all distances, times, and causal structure. \\
\hline
\textbf{Source--effect link} &
Sources (masses) are plugged into chosen formulas for $n$ and $\mathbf{u}_g$. Relationship is phenomenological, not derived from a covariant field equation. &
Einstein field equations $G_{\mu\nu} = \frac{8\pi G}{c^4}T_{\mu\nu}$ link stress--energy to curvature in a self‑consistent way. \\
\hline
\textbf{Motion of matter} &
Massive bodies: Newtonian $N$‑body integration in flat space. &
Massive bodies: timelike geodesics in curved spacetime. \\
\hline
\textbf{Motion of light} &
Fermat’s principle in a medium: rays bend toward higher $n$; perpendicular component of $\nabla n$ changes direction. Optional advection by $\mathbf{u}_g$. &
Null geodesics in curved spacetime; bending emerges from the metric. \\
\hline
\textbf{Frame‑drag analogue} &
Explicit $\mathbf{u}_g$ term advects rays/particles. &
Off‑diagonal metric terms (e.g., Kerr metric) produce frame‑dragging naturally. \\
\hline
\textbf{Gravitational waves} &
Added as explicit sinusoidal modulations of $n$ (TT‑like), not emergent. &
Oscillations of spacetime curvature itself, propagating at $c$, derived from Einstein equations. \\
\hline
\textbf{Time evolution} &
Medium changes because sources move; $n$ and $\mathbf{u}_g$ recomputed each tick. &
Metric evolves according to coupled PDEs with matter dynamics. \\
\hline
\textbf{Computational load} &
Lightweight: direct evaluation of $n$, $\nabla n$, $\mathbf{u}_g$ and ray/body integration. Modular toggles for effects. &
Heavy: solve nonlinear PDEs for $g_{\mu\nu}$ in 3+1D; numerical relativity for general cases. \\
\hline
\textbf{Strengths} &
Intuitive optical analogy, modular, interactive, pedagogically clear, fast to simulate. &
Physically complete within its domain, matches all tested regimes, built‑in Lorentz invariance. \\
\hline
\textbf{Limitations} &
Not derived from a covariant theory; no guarantee in strong‑field/high‑velocity regimes; wave/drag effects are modelled by hand. &
Computationally expensive; less intuitive for newcomers; exact solutions rare. \\
\hline
\end{tabularx}
\caption{Functional comparison between the pushing‑medium model and General Relativity.}
\end{table}

